\documentclass[t, 10pt]{beamer}
%%\documentclass[t,handout]{beamer}

\usepackage{graphicx}
\usepackage{epsfig}
\usepackage{psfrag}
\usepackage[english]{babel}
\usepackage{color}
%Mathematics packages
\usepackage{amsmath}
\usepackage{mathrsfs}
\usepackage{amsfonts}

\usepackage{enumerate}


\graphicspath{{./images/}} % Figures path - used in graphicx

\selectcolormodel{cmyk}

\mode<presentation>

%THEMES - Please refer to these chapters in the beamer documentation.
% Presentation themes : Chapter 15
% Color themes : Chapter 17
% Font themes : Chapter 18


\usetheme{Pittsburgh}
\usecolortheme{orchid}
\usefonttheme{default}

%---------------------------Title frame definition------------------------------------- 

\title{Behavior Driven Development}
\author [Chris]{Christopher C. Lamb}
\institute[University of New Mexico]{
\inst {}Department of Electrical and Computer Engineering\\
University of New Mexico}
\date{June 13, 2011}
\titlegraphic{
\begin{figure} 
\includegraphics[width = 7cm]{UNM}
\end{figure}}

% Delete this, if you do not want the table of contents to pop up at
% the beginning of each subsection:
%\AtBeginSubsection[]
%{
%  \begin{frame}<beamer>
%    \frametitle{Outline}
%     \tableofcontents[currentsection,currentsubsection]
%  \end{frame}
%}

\begin{document}

\begin{frame}
\titlepage
\end{frame}

% This command will make the logo appear on all frames excluding the title frame.
\logo {\includegraphics[width = 2.5cm]{UNM}}

\begin{frame}
\frametitle{Outline}
\tableofcontents 
\end{frame}

\section{What is Behavior Driven Development?}

\begin{frame}
\frametitle{What is Behavior Driven Development?}
\textbf{Behavior Driven Development (BDD) is a second-generation, outside-in, pull-based, multiple-stakeholder, multiple-scale, high-automation, agile methodology. It describes a cycle of interactions with well-defined outputs, resulting in the delivery of working, tested software that matters.}
\newline
\textit{--- Dan North, London, 2009}
\newline
\newline
\newline
\pause
Basically, this means that it is driven by specific, quantifiable value associated by produced code.  The code is produced with an eye toward \textit{verifiable quality}, and as much of the quality strategy as possible is automated.  This implies that we have automated tests and automated builds.  BDD builds on \textit{Test Driven Development} (TDD), which in turn builds on disciplined unit testing.  We'll cover all of this today.
\end{frame}

\section{Unit Testing}

\begin{frame}
\frametitle{So what is this testing of units?}
Unit testing has been around for 40 years or so.  Basically, you test the smallest units of code that are assembled into your system.  People used to do it with mainframes.  Automated testing's been around that long too, but people used to write their own test frameworks for this kind of thing.  About 15 years ago, Things changed.
\pause
\newline
\newline
Kent Beck and Erich Gamma created JUnit, a unit testing framework to support the development methodology they were evangelizing (e.g. eXtreme Programming).  Both of them are still around, and both were and are very influential.  Now we test classes and methods.
\newline
\newline
This established a de-facto standard for unit test frameworks now referred to as \textit{xUnit}.
\end{frame}

\begin{frame}
\frametitle{Why do we do it?}
Developing software is hard.  It's getting harder and harder as time goes by.  Software systems are becoming:
\begin{itemize}
\item \textit{More Complex} \\
Ten years ago distributed systems were hard.  Mobile computing was just starting.  Now everything's distributed, cloud computing is widely used by developers, mobile computing ubiquitous.  This all depends on layers upon layers of usually third-party libraries.
\pause
\item \textit{Less Trustworthy} \\
So we use all these new libraries to create software, how do we know that it's trustworthy? that we're using it correctly? When we need to change our code base, do we know we can do it without causing other problems?
\end{itemize}

\end{frame}

\begin{frame}
\frametitle{Why do we do it?}
This is much better than the way we used to develop and test software.
\newline
\newline
\newline
\pause
Back in the old days, most projects had dedicated testing teams, with legions of testers.  These testers would test the system through some kind of user interface, something they built or one of the developers built, and then they would test the system according to some huge test plan that outlined thousands of steps they needed to go through to test the software.
\newline
\newline
\newline
\pause
\textbf{This cost a fortune, and didn't work.}
\end{frame}

\begin{frame}
\frametitle{Why do we do it?}
Just writing the test plan was a horrible, expensive experience.  These things were huge, and took lots of thought and time.  Then you needed an army to execute it, and it took forever, all of which cost \$\$\$.  And how can you test error conditions in your communication layer when it is two libraries removed from the user interface?
\newline
\newline
\newline
\pause
Answer: you can't, not through the UI.
\newline
\newline
\newline
\textbf{Enter unit tests.}
\end{frame}

\begin{frame}
\frametitle{Why do we do it?}
Unit testing solves many of these problems.  Unit tests are:
\begin{itemize}
\item \textit{Automated} --- Huge test suites run in seconds, and they can be automated to run whenever you want them to.  They can be configured to run whenever a developer checks \textit{any changes whatsoever} into a code base.
\item \textit{Functional} --- The tests not only test the code, they serve as examples of how the code is to be used.
\item \textit{Flexible} --- You can now test that hard-to-reach communication layer by building unit tests right into that code.
\item \textit{Aggregateable} --- You can pull together unit tests for all of your code into \textit{test suites}.
\end{itemize}
\end{frame}

\begin{frame}
\frametitle{How do we do it?}
Generally, you will test individual units of functionality in your code.  Thing classes, methods, functions.  Unit tests are self contained, and not dependent on any single runtime environment.  To do this you usually use mocks or stubs, which \textit{emulate} dependencies.  To do this, you need to design your code to be testable, usually using a pattern called \textit{Dependency Injection}.
\newline
\newline
\pause
We can use unit testing frameworks to support a variety of other testing methods as well.  You can use them for:
\begin{itemize}
\item \textit{Functional testing} --- no mocks, tied to runtime environment
\item \textit{Integration testing} --- testing more than one unit; e.g. multiple layers or components
\end{itemize}
\end{frame}

\begin{frame}
Unit testing allows more complete testing, less complexity, and better software through:
\begin{itemize}
\item documentation of programmatic use
\item better design, forces use in more than one domain, design for test
\item more agility, easier to change your software
\item ease of later integration
\end{itemize}
Now keep in mind, unit tests aren't free.
\newline
\newline
\pause
They take time to write, and if you're not careful you can create \textit{fragile tests}, which break easily when the code is changed, leading you to spend an inordinate amount of time altering your test suite.  These are the two most sensible arguments people use against unit testing.  Less sensible arguments include the \textbf{my code doesn't need them} or the \textbf{I already know how to do my job} arguments.
\end{frame}

\section{Test Driven Development}

\begin{frame}
\frametitle{So what is Test Driven Development?}
Well, it's another way to build software.
\newline
\newline
\pause
Remember, even if unit testing has been around for a while, it was primarily a niche practice.  For the most part, industry was using large test teams, understood that they didn't work well, but they knew how to do it.  So that's what people did.  When unit testing started to gain more mainstream acceptance, the flexibility of the practice (and the xUnit frameworks) allowed developers to change how they did their jobs, to enable them to produce better code.
\newline
\newline
\pause
Some renegade developers started \textit{writing tests for code that didn't exist}, and then they would create code so the tests would pass.
\newline
\newline
Ergo, \textbf{Test Driven Development} was born.
\end{frame}

\begin{frame}
\frametitle{How does it work?}
You use a simple algorithm:



\end{frame}

\begin{frame}
\frametitle{}
\end{frame}

\end{document}

