\documentclass[10pt,letterpaper]{letter}
\usepackage[utf8]{inputenc}
\usepackage{amsmath}
\usepackage{amsfonts}
\usepackage{amssymb}
\begin{document}
Welcome everyone and thank you for joining us today.   I'm honored to be the graduate speaker for today's convocation ceremony.  And a heartfelt thank you to the staff and faculty of the electrical and computer engineering department for all their support through my PhD program, and the computer science department for my master's degree.  I'd especially like to thank my PhD advisor, Greg Heileman, my Master's advisor, George Luger, and especially my wife for her support throughout the years.

So these kinds of talks can be difficult to put together.  Lots of people have pretty strong ideas of what you should discuss.  At the end of the day, it boils down to me trying to tell you something meaningful you can use after you graduate, and do it in 300 seconds or less.  Preferably more like 180 seconds if I can.

So I have two things I'd like to touch on today.  The first one is about engineering as a field, and the second about how you can succeed in engineering as a career.

So to begin with, you need to accept that you'll need to do your homework for the rest of your career.  Fortunately, all your homework throughout your engineering curriculum has prepared you well for this --- you're already used to it.  The bottom line is that engineering --- like law, consulting, or the military --- is an up-or-out profession.  You're not going to be able to do the same thing you're doing now in five years. Honestly, not even two.  This isn't limited to fields like computer science or electrical engineering either.  Civil engineers, mechanical engineers, we all need to accept this.  This doesn't mean that you need to find a job in management after a few years, but it does mean that you need to continue to be able to produce more and more value as you advance in your career.  And to do this, you need to do your homework.  You need to know what to study and what to work on so you can produce that value.  Figuring this out takes time and effort; you need to think hard about what you want to do next, and then decide how to get there from where you are.  The PhD I'm graduating with today helped me.  Your path will be different.

Once you've figured out where you want to go and how to get there, you need to stay focused and not quit.  This takes a certain amount of grit, especially as you get older.  You'll find that not everybody is going to support you in your goals, for reasons they might not even understand.  So far in my life, I've had people tell me that my goals were unachievable, or that they weren't interested in working with me, or that my work was sub-par.  I guarantee all this will happen to you too.  You'll have distractions, you'll have conflicting goals, you'll wonder if all the hard work is really worth it.  You need to have done your homework so you can have confidence that you're doing the right thing and that despite all that life may put in your way you can cowboy up and take care of business.  Sound like a bit much? it's not, not really.  If I can do it, you can too.

Engineering is one of the few fields where you really can make a tangible difference in the world.  It can lead to an engaging, thought-provoking career, and can provide opportunities just not available in other fields.  I've been doing this for almost 20 years now, and will continue for at least 20 more.  I wish you all the best and a bright future.

Thank you.
\end{document}