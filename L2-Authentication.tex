%%\documentclass[t]{beamer}
\documentclass[t,handout]{beamer}

\usepackage{graphicx}
\usepackage{pgf}
\usepackage{epsfig}
\usepackage{psfrag}
\usepackage[english]{babel}
\usepackage{hyperref}

\usepackage{listings}
\usepackage{courier}
\usepackage{color}

\lstset{
	language=Ruby,
	basicstyle=\footnotesize\ttfamily\color{black},
	commentstyle = \footnotesize\ttfamily\color{red},
	keywordstyle=\footnotesize\ttfamily\color{blue},
	stringstyle=\footnotesize\ttfamily\color{black},
%	columns=fixed,
%	numbers=left,    
	numberstyle=\tiny,
	stepnumber=1,
	numbersep=5pt,
	tabsize=1,
	extendedchars=true,
	breaklines=true,            
	frame=b,         
	showspaces=false,
	showtabs=true,
	xleftmargin=6pt,
	framexleftmargin=6pt,
	framexrightmargin=2pt,
	framexbottommargin=4pt,
	showstringspaces=false      
}

\lstloadlanguages{
         Ruby,HTML
}
 %\DeclareCaptionFont{blue}{\color{blue}}

\newcommand{\asgn}{\mbox{$\bf\leftarrow$ }}  %pseudocode assignment stmt
\newcommand{\comt}{\mbox{$\; \rhd$ \ }}  %pseudocode comment

\graphicspath{{images/}} % Figures path - used in graphicx

%\selectcolormodel{cmyk}

\mode<presentation>

\usetheme{Warsaw}

\title[Module 5, Lecture 2]{ECE 495/595 -- Web Architectures/Cloud Computing}
\subtitle{Module 5, Lecture 2: Web Application Security -- Authentication}

\author[\copyright \ 2011-12 G. L. Heileman]{Professor G.L. Heileman}
\date{}
\institute[University of New Mexico]{\large University of New Mexico}

\titlegraphic{
\begin{figure} 
\hspace{2cm}
\includegraphics[width = 4in]{ECE-UNM-logo.pdf}
\end{figure}}

\begin{document}

\begin{frame}
\titlepage
\end{frame}

% This command will make the logo appear on all frames excluding the title frame.
\logo{\includegraphics[width = 0.75in]{UNM.pdf}}

%---------------------------------------------------------------------- frame -
\begin{frame}\frametitle{Web Application Security}
\begin{itemize}
\item In the previous lecture we considered the basic notions behind user authentication in web applications.
 \pause
 \item In this lecture we'll consider how to build your own user authentication into a Rails application. \pause
 This will involve adding user authentication to the blog application, and will include a consideration of the various ways you can manage session information within a rails application.
 \pause
 \item Next we'll discuss some of the off-the-shelf authentication packages that can be incorporated into a Rails application.
 \pause
 \item Which approach you take depends upon the requirements of your application, how much customization you need to
 perform, etc. --- whatever the case,  building authentication from scratch will help you to better understand 
 authentication packages.
\end{itemize}
\end{frame}

%---------------------------------------------------------------------- frame -
\begin{frame}[fragile]\frametitle{Authentication from Scratch -- Adding Users}
 \begin{itemize}
   \item  The first thing we need to do is get rid of the HTTP-based authentication that is currently in application.  I.e., delete the following line in \lstinline!posts\_controller.rb!:
\begin{lstlisting}[frame=none]
# app/controllers/posts_controller.rb
http_basic_authenticate_with :name => 'admin',
   :password =>'secret', :except =>[:index, :show]
\end{lstlisting}
\pause
Next, we need to add support for users in the blog application.  We won't generate a scaffold, because we don't need all of the functionality. \pause
Users will login using their email address (to differentiate users) and a password, so run the following:~\\
\begin{lstlisting}[frame=none]
$ rails g model user email:string 
      password_hash:string password_salt:string
\end{lstlisting}
\pause
In order to create new users, we'll include the new action in the the users\_controller:
\begin{lstlisting}[frame=none]
$ rails g controller users new 
$ rake db:migrate
\end{lstlisting}
 \end{itemize}
\end{frame}

%---------------------------------------------------------------------- frame -
\begin{frame}\frametitle{Authentication from Scratch -- Sign Up}
\vspace*{-0.125in}
The form for signing up, {\tt app/views/users/new.html.erb}:
{\tt\footnotesize \begin{tabbing} **\=**\=**\=**\=**\=**\= \kill
\> <h1>Sign Up</h1> \\ 
\> <\%= form\_for @user do |f| \%> \\
\>\>  <\% if @user.errors.any? \%> \\
\>\>\>    <div class="error\_messages"> \\
\>\>\>\>      <h2>Form is invalid</h2> \\ 
\>\>\>\>      <ul> \\
\>\>\>\>\>        <\% for message in @user.errors.full\_messages \%> \\
\>\>\>\>\>\>          <li><\%= message \%></li> \\
\>\>\>\>\>        <\% end \%> \\
\>\>\>\>      </ul> \\
\>\>\>    </div> \\
\>\>  <\% end \%> \\
\>\>  <p> <\%= f.label :email \%><br /> \\
\>\>\>\>    <\%= f.text\_field :email \%> </p> \\
\>\>  <p>  <\%= f.label :password \%><br /> \\
\>\>\>\>    <\%= f.password\_field :password \%> </p> \\
\>\>  <p> <\%= f.label :password\_confirmation \%><br /> \\
\>\>\>\>    <\%= f.password\_field :password\_confirmation \%> </p> \\
\>\>  <p class="button"><\%= f.submit  "Sign Up" \%></p> \\
\> <\% end \%>
\end{tabbing}}
\end{frame}

%---------------------------------------------------------------------- frame -
\begin{frame}\frametitle{Authentication from Scratch -- Sign Up}
\vspace*{-0.1in}
{\small
 \begin{itemize}
  \item Notice the if statement that will be executed if there are any validation errors associated with the model.
  \pause
  \item Right now, if you start up the server, and navigate to {\tt http://localhost:3000/users/new}, you'll get an error because we have not handed the form a user object to work with.  \pause   In addition the {\tt create} action that will be called when the form is submitted has not been defined.  We take care of both of these actions on the next slide. \pause
  \item One more thing we should fix.  When we generated the users controller, it added the following to {\tt config/routes.rb}:
  {\tt\footnotesize \begin{tabbing} **\=**\=**\=**\=**\=**\= \kill
   \>  get "users/new"
   \end{tabbing}}
  That's why the URL above works. \pause But this doesn't look very professional. To fix this, replace the line above with;
    {\tt\footnotesize \begin{tabbing} **\=**\=**\=**\=**\=**\= \kill
  \>  get "sign\_up" => "users\#new", :as => "sign\_up" \\
  \> root :to => "posts\#index" \\
  \> resources :users
   \end{tabbing}}
   \pause
   This gives us the URL {\tt\footnotesize http://localhost:3000/sign\_up}
 \end{itemize}}
\end{frame}

%---------------------------------------------------------------------- frame -
\begin{frame}\frametitle{Authentication from Scratch -- Sign Up}
{\small
\begin{itemize}
 \item Edit users\_controller.rb, defining the {\tt new} and {\tt create} actions:
{\tt\scriptsize \begin{tabbing} \=**\=**\=**\=**\=**\=** \kill
 class UsersController < ApplicationController \\ 
\>  def new \\
\>\>   @user = User.new \\
\>  end \\ \\
\>  def create \\
\>\>    @user = User.new(params[:user]) \\
\>\>    if @user.save \\
\>\>\>      redirect\_to root\_url, :notice => "Successfully signed up!" \\
\>\>   else \\
\>\>\>      render "new" \\
\>\>    end \\
\>  end \\
 end
\end{tabbing}}   
\pause
 \item Now, when you navigate to ~\\
  \ \ \ \ {\tt http://localhost:3000/sign\_up}~\\
  you should see the form.  
  \pause
  Notice how the password and password\_confirmation fields don't display the text typed into them.
 \end{itemize}}
\end{frame}   

%---------------------------------------------------------------------- frame -
\begin{frame}[fragile]\frametitle{Authentication from Scratch -- Sign Up}
\vspace*{-0.1in}
{\small
 \begin{itemize}
  \item What happens when you try to sign up a new user?  \pause You get an error: ~\\
   {\footnotesize\tt Can't mass-assign protected attributes: password, password\_confirmation} ~\\ \pause
  Why?  \pause Well, think back to how we generated the model, \pause there are no password or password confirmation attributes in our user model.
  \pause
  \item We want the model itself to have a password attribute, but we don't want the password to be saved in the database (remember, we're  going to store a hash of the password instead).  \pause  To accomplish this add the following to {\tt app/models/user.rb}:
  
\begin{lstlisting}[frame=none,language=Ruby,basicstyle=\scriptsize\ttfamily\color{black},]
class User < ActiveRecord::Base
  attr_accessible :email, :password_hash, :password_salt, :password, :password_confirmation
  attr_accessor :password
  validates :email, :presence => true, 
      :uniqueness => {:case_sensitive => false}
  validates :password, :confirmation => true  
end
\end{lstlisting}
   \pause
   \item  Note: I took the time to add some validations too! 
 \end{itemize}} 
\end{frame}

%---------------------------------------------------------------------- frame -
\begin{frame}\frametitle{Authentication from Scratch -- Sign Up}
{\small
 \begin{itemize}

   \item Now, when you navigate to~\\
    \ \ \ \ {\tt http://localhost:3000/sign\_up} ~\\
    you should be able to sign up new users. \pause You should also check that the validations are working.
   \pause
   \item Sign some up, and check the database (using SQLite Manager), the email is saved to the database, but the {\tt password\_hash} and {\tt password\_salt} are empty.
   \pause
   \item In order to make the root route work, you need to remove the file {\tt ./public/index.html}.
   \pause
   \item We'll add the gem {\tt bcrypt} and use it's one-way hash function.  In your Gemfile add the line:~\\
   \ \ \ \ {\tt gem 'bcrypt-ruby', :require => 'bcrypt'}~\\
   and run~\\
    \ \ \ \ {\tt \$ bundle install}
    \pause
    \item Now add a before filter to the {\tt User} model that takes care of encrypting the password (next slide).
 \end{itemize}}
\end{frame} 

%---------------------------------------------------------------------- frame -
\begin{frame}[fragile]\frametitle{Authentication from Scratch -- Sign Up}
\begin{lstlisting}[frame=none,language=Ruby,basicstyle=\scriptsize\ttfamily\color{black},xleftmargin=2pt]
class User < ActiveRecord::Base
 # accessor declarations and validations
  
  before_save :encrypt_password
    
  def encrypt_password
    if password.present?
      self.password_salt = BCrypt::Engine.generate_salt
      self.password_hash = BCrypt::Engine.hash_secret(password,password_salt)
    end
  end
  
end
\end{lstlisting}
\end{frame}   

%---------------------------------------------------------------------- frame -
\begin{frame}\frametitle{Authentication from Scratch -- Log In}
 \begin{itemize}
   \item When you navigate to {\tt http://localhost:3000/sign\_up}, you should be able to sign up new users, and you should see the encrypted password and 
   password salt stored in the database as well.
      \pause
   \item We're halfway there --- users are now able to sign up, but they can't log in yet.  \pause To take care of this, we'll need to write the code that manages user sessions.
   \pause
   \item Let's start by creating a controller for sessions:~\\
   \ \ \ \ \ {\tt\small \$ rails g controller sessions new} ~\\
   \pause
   Once again, we need to supply a form to work with the {\tt new} action in the sessions controller.  
   \pause 
   This is the form that will be used for signing in.
   \pause
   \item Because there is no session model, we will use the {\tt form\_tag} method, rather than the {\tt form\_for} method.  We simply want the form to POST to the {\tt sessions\_path}, which is the session controller's {\tt create} action.
  \end{itemize}
\end{frame} 

%---------------------------------------------------------------------- frame -
\begin{frame}\frametitle{Authentication from Scratch -- Log In}
\begin{itemize}
 \item In /app/views/sessions/new.html.erb, write:
{\tt\footnotesize \begin{tabbing} **\=**\=**\=**\= \kill
\> <h1>Log in</h1> \\ 
\> <\%= form\_tag sessions\_path do \%> \\
\>\>   <p>  <\%= label\_tag :email \%><br /> \\
\>\>\> \>    <\%= text\_field\_tag :email, params[:email] \%>  </p> \\
\>\>    <p> <\%= label\_tag :password \%><br /> \\
\>\>\> \>     <\%= password\_field\_tag :password \%>  </p> \\
\>\>    <p class="button"><\%= submit\_tag "Log In" \%></p> \\
\> <\% end \%> \\
\end{tabbing}}
\pause
\item We should also modify the routes.  In {\tt config/routes.rb} get rid of: ~\\
{\tt\footnotesize \begin{tabbing} **\=**\=**\=**\= \kill
\> {get "sessions/new"} 
\end{tabbing}}
and add:
{\tt\footnotesize \begin{tabbing} **\=**\=**\=**\= \kill
\> get "log\_in" => "sessions\#new", :as => "log\_in" \\
\> get "log\_out" => "sessions\#destroy", :as => "log\_out" \\
\> resources :sessions
\end{tabbing}}
\end{itemize}
\end{frame} 

%---------------------------------------------------------------------- frame -
\begin{frame}[fragile]\frametitle{Authentication from Scratch -- Authenticating}
\vspace*{-0.1in}
{\small
We need to authenticate the user when they try to log in.  To do this, add the following method in
{\tt /app/models/user.rb}:
\begin{lstlisting}[frame=none,language=Ruby,basicstyle=\scriptsize\ttfamily\color{black}]
def self.authenticate(email, password) 
   user = find_by_email(email)
   if user && user.passwordhash == BCrypt::Engine.hash_secret(password, user.password_salt) 
      user
   else 
      false
   end 
end
\end{lstlisting}
 \begin{itemize} 
   \pause
   \item This method will be invoked through a User object whenever a new session needs to be created.
   \pause
   \item The user-supplied email  is used to find a user object. \pause Then the user-supplied password, along with the stored password salt for the user, are run 
   through the same one-way hash function used to create the stored password hash.  \pause If the computed password hash matches the stored password hash, 
   the user is authenticated.
 \end{itemize}}
\end{frame}  

%---------------------------------------------------------------------- frame -
\begin{frame}[fragile]\frametitle{Authentication from Scratch -- Log In}
{\small
The create method in {\tt app/controllers/sessions\_controller.rb}:
\begin{lstlisting}[frame=none,language=Ruby,basicstyle=\scriptsize\ttfamily\color{black},	xleftmargin=0pt]
def create
    @user = User.new(params[:user])
    if @user.save
      redirect_to root_url, :notice => "Successfully signed up!" 
    else
      render "new"
    end
end
\end{lstlisting}
\pause
\begin{itemize}
 \item If a user is returned from the authentication method (authentication successful), then the user's id is stored in the Rails session hash.
 \pause
 \item If a user object is not returned (authentication fails), then a flash message is created, and the application redirects to the new method (which has been mapped to log\_in) in the sessions controller.
\end{itemize}
}
\end{frame}

%---------------------------------------------------------------------- frame -
\begin{frame}\frametitle{Authentication from Scratch -- Log Out}
\begin{itemize}
\item  We also need to handle logging out.  Add a {\tt destroy} action to 
{\tt app/controllers/sessions\_controller.rb}:
{\tt\footnotesize \begin{tabbing} **\=**\=**\=**\= \kill
\>   def destroy \\
\>\>     session[:user\_id] = nil \\
\>\>    redirect\_to root\_url, :notice => "Logged out!" \\
\>  end
\end{tabbing}}
\pause
 \item We previously set up the {\tt log\_out} route to use this action.
 \pause
 \item After destroying the session, the application routes to the {\tt root\_url}.
\end{itemize}
\end{frame}

%---------------------------------------------------------------------- frame -
\begin{frame}\frametitle{Authentication from Scratch -- Log In}
{\small
\begin{itemize}
\item To display the flash messages, add the following to {\tt app/views/layouts/application.html.erb}, in the body, just before the {\tt <\%= yield \%>} statement.
{\tt\footnotesize \begin{tabbing} **\=**\=**\=**\= \kill
\>  <\% flash.each do |name, msg| \%> \\
\>\> <\%= content\_tag :div, msg, :id => "\#\{name\}" \%> \\
\>  <\% end \%>
  \end{tabbing}}
  \pause 
 \item If you'd like to use the ActiveRecord session store instead, you simply need to do the following
{\tt\footnotesize \begin{tabbing} **\=**\=**\=**\= \kill
\> \$ rake db:sessions:create \\
\> \$ rake db:migrate
\end{tabbing}}
This creates a table for sessions in the database.~\\
\pause
Then, you need to tell Rails to use it via a setting in {\tt config/initializers/session\_store.rg}:
{\tt\footnotesize \begin{tabbing} **\=**\=**\=**\= \kill
\> \$ MyApplication::Application.config.session\_store \\
\ \ \ \ \ \ \ \ \ \ \ \ \ \ \ \ \ \ \ \ \ \ \ \ \ \ \ \ \ \ \ \ \ \ \ \ :active\_record\_store 
\end{tabbing}}
 \end{itemize}}
\end{frame}
  
%---------------------------------------------------------------------- frame -
\begin{frame}[fragile]\frametitle{Authentication from Scratch -- Adding Links}
{\small
\begin{itemize}
\item Finally, let's add some links to the application that let a user sign-up and log in/out from any page in our blog application.
\pause
\item Place the following before the {\tt <\%= yield \%>} statement. in  {\tt app/views/layouts/application.html.erb}:
\begin{lstlisting}[frame=none,language=HTML,basicstyle=\scriptsize\ttfamily\color{black}]
<div id="user_nav">
  <% if current_user %>
      Logged in as <%= current_user.email %>
      <%= link_to "Log out", log_out_path %> 
  <% else %>
      <%= link_to "Sign up", sign_up_path %> or
      <%= link_to "Log in", log_in_path %> 
  <% end %>
</div>
\end{lstlisting}
\item Because this div was placed in the {\tt application.html.erb} file, this HTML will be executed prior to executing any other view code.
\pause
\item If there's a {\tt current\_user}, one set of links will be displayed, and if there's not, a different set of links is displayed.
 \end{itemize}}
\end{frame}  

%---------------------------------------------------------------------- frame -
\begin{frame}[fragile]\frametitle{Authentication from Scratch -- Adding Links}
\begin{itemize}
\item However, the {\tt current\_user} is not defined yet.  To do that,  modify  {\tt app/controllers/application\_controller.rb} to look like:
\begin{lstlisting}[frame=none,language=Ruby,basicstyle=\scriptsize\ttfamily\color{black}]
class ApplicationController < ActionController::Base
  protect_from_forgery
  
  private
  
  def current_user
    @current_user ||= User.find(session[:user_id]) if session[:user_id]
  end
  
  helper_method :current_user
end
\end{lstlisting}
  \pause
    \item {\tt current\_user} is now a helper method that will be made available to every controller.
  \end{itemize}
\end{frame}  

%---------------------------------------------------------------------- frame -
\begin{frame}\frametitle{Authentication from Scratch -- Summary}
{\small
\begin{itemize}
 \item Right now, we're sending user credentials in the clear.  To fix this, add the {\tt force\_sll} class method to {\tt user\_controller.rb} and {\tt session\_controller.rb}.
 \pause
 \item We should probably also add some additional validations. E.g., you might consider
  \begin{itemize}
   \item[--] Making sure the email provided looks like an email address.
   \pause
   \item[--] Ensuring that passwords are strong -- right now, you can supply a password at sign-up that is a single character.
   \end{itemize}
  \pause
  \item Lastly, in real applications, we need to handle the situation of users forgetting their username or password, logging user activity, etc.
  \pause
  \item Related to this notion, why are we saying ``Invalid email or password" when user enters either an invalid email (on not in our database) or password?~\\
  Why don't we tell them which one is invalid?
 \end{itemize}}
\end{frame}  

%---------------------------------------------------------------------- frame -
\begin{frame}\frametitle{Authentication from Scratch -- Summary}
{\small
\begin{itemize}
\item That's it, we now have basic user authentication integrated into our blog application.
\pause
\item However, we haven't used it for anything yet.  \pause Specifically, we'd like to use this authentication to restrict users from accessing certain resources --- this is referred to as \textcolor{blue}{access control}.
\pause
\item In fact, we'd like to assign roles to users, e,g., administrator, moderator, user, and then restrict access according to the users role --- this is referred to as \textcolor{blue}{role-based access control}.  We'll cover that in the next lecture.
\pause
\item In addition, since we now have a good idea as to how authentication works, we'll add access control using on top of an existing Rails authentication framework --- this is the way you will normally bring authentication/access control into your applications.
\pause
 \end{itemize}}
\end{frame}  

\end{document}

\item The most popular Rails authentication frameworks include: Authlogic, Restful Authentication, Clearance, Devise and Sorcery.


